%=================================================================
%\section{Unsupervised expansion of ad-hoc abbreviations in EHR narratives}
\section{Goal}

%-----------------------------------------------------------------
%\subsection{Goal}

\begin{frame}
	\frametitle{1\textsuperscript{st} problem: abbreviation expansion}
	\begin{exampleblock}{Original}
	3. St.p. TE eines exulz. sek.knot.SSM (C43.5) li Lab. majus.\\
	Level IV, 2,42 mm Tumordurchm.
	\end{exampleblock}
	
	\pause
	
	\begin{exampleblock}{Expanded}
	Status post Totalexzision eines exulzerierenden sekundär knotigen superfiziell\\
	spreitenden Melanoms (C43.5) linkes Labium majus.\\
	Level Vier, 2,42 Millimeter Tumordurchmesser.
	\end{exampleblock}
	
	\pause
	
	\begin{exampleblock}{English translation}
	3. History of total excision of an exulcerated secondarily nodular superficially\\
	spreading melanoma (C43.5) of the outer left labia.\\
	Level 4, tumor diameter 2.42mm.
	\end{exampleblock}
\end{frame}

%\begin{frame}
%	\frametitle{Goal}
%	\begin{itemize} \myspacing
%		\item Unsupervised expansion of ad-hoc abbreviations in clinical narratives\footnotemark
%		\begin{itemize}
%			\item Unsupervised: no time/resources to train the system
%			\item Expansion: solve ambiguities
%			\item Ad-hoc: not in a dictionary
%			%\item EHR: Electronic Health Record
%			\item Narratives: free-text, no extra information available
%		\end{itemize}
%	\end{itemize}
%	\footnotetext[1]{Contribution to CBmed Project 1.2 (IICCAB)}
%\end{frame}
